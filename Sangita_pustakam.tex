\documentclass[12pt]{article}

%%%%%%%%%%%%%%%%%%%%%%%%%%%%%%%%%%%%%%%%%%%%%%%%%%%%%%%%%%%%%%

\usepackage[a4paper, total={7in, 10in}]{geometry}
\usepackage{fontspec}
\usepackage{polyglossia}
\usepackage{amsmath}
\usepackage{stackrel}
\usepackage{longtable}
\usepackage{adjustbox}

\usepackage[table, dvipsnames]{xcolor}
\definecolor{Gray}{gray}{0.9}
\definecolor{Saffron}{HTML}{FF4500}

\usepackage{hyperref}
\hypersetup{
    colorlinks=true, %set true if you want colored links
    linktoc=all,     %set to all if you want both sections and subsections linked
    linkcolor=blue,  %choose some color if you want links to stand out
}

\usepackage{tikz}
\usetikzlibrary{calc}
\usetikzlibrary{decorations.pathmorphing}
\usetikzlibrary{decorations.text}
\makeatletter
\renewcommand{\@seccntformat}[1]{}
\makeatother
  
%%%%%%% Anukramanika page %%%%%%%%%%%%%%%

\makeatletter
\renewcommand\tableofcontents{%
    \@starttoc{toc}%
}
\makeatother

\makeatletter
\let\latexl@section\l@section
\def\l@section#1#2{\begingroup\let\numberline\@gobble\latexl@section{#1}{#2}\endgroup}
\makeatother

\makeatletter
\let\latexl@subsection\l@subsection
\def\l@subsection#1#2{\begingroup\let\numberline\@gobble\latexl@subsection{#1}{#2}\endgroup}
\makeatother


%%%%%%%%%%%%%%%%%%%%%%%%%%%%%%%%%%%%%%%%

\setmainlanguage{sanskrit} 

\newfontfamily\shobhika[Script=Devanagari,Mapping=devanagarinumerals]{Shobhika}
\newfontfamily\devanagarifont[Script=Devanagari,Mapping=devanagarinumerals]{Sanskrit 2003}

\title{\begin{sanskrit} \textbf{सङ्गीत पुस्तकम्} \end{sanskrit}}
\date{}
\author{\begin{sanskrit} \textbf{गुरु : श्री श्रीधर नरसिह्मन्}\end{sanskrit}}
\usepackage{titling}
\renewcommand\maketitlehooka{\null\mbox{}\vfill}
\renewcommand\maketitlehookd{\vfill\null}

%% Defining functions to enter tara and mandhara swaras %%%%%%

\newcommand{\tar}[1]{\stackrel[]{*}{\text{\begin{sanskrit} #1 \end{sanskrit}}}}

\newcommand{\man}[1]{\stackrel[\textrm{*}]{}{\text{\begin{sanskrit} #1 \end{sanskrit}}}}

%% Index for ragas, composers and kritis

\usepackage{imakeidx}
\usepackage{blindtext}
\makeindex[name=ragas, columns=1, title=\begin{sanskrit}रागसूचि\end{sanskrit}]
\makeindex[name=composers, columns=1, title=\begin{sanskrit}रचयिता सूचि\end{sanskrit}]

% Section and subsection title format

\usepackage{titlesec}
\titleformat{\section}[block]{\color{Saffron}\Large\bfseries\filcenter}{}{1em}{}
\titleformat{\subsection}[hang]{\color{Saffron}\bfseries\filcenter}{}{1em}{}


\usepackage[utf8]{inputenc}

%%%%%%%%%%%%%%%%%%%%%%%%%%%%%%%%%%%%%%%%%%%%%%%%%%%%%%%%%%%%%%%%%%%%%%%%%%%%%%%%%%%%%%%%%%%%%%
%%%%%%%%%%%%%%%%%%%%%%%%%%%%%%%%%%%%%%%%%%%%%%%%%%%%%%%%%%%%%%%%%%%%%%%%%%%%%%%%%%%%%%%%%%%%%%
\begin{document}
\begin{sanskrit} %%%%%%%%%%%%%%%%%%%%%%%%%%%%%%%%%%%%%%%%%%%%%%%%%%%%%%%%%%%%%%%%%%%%%%%%%%%%%%

%%%%%%%%%%%%%%%%%%%%%%%%%%%%%%%%%%%%%%%%%%%

\begin{titlingpage}
\pagenumbering{gobble}
\begin{center}
\begin{tikzpicture}[overlay,remember picture]
    \draw [line width=1mm]
        ($ (current page.north west) + (1cm,-1cm) $)
        rectangle
        ($ (current page.south east) + (-1cm,1cm) $);
    \draw [line width=0.5mm]
        ($ (current page.north west) + (1.2cm,-1.2cm) $)
        rectangle
        ($ (current page.south east) + (-1.2cm,1.2cm) $);
\end{tikzpicture}
\vspace*{\fill}\\

 \textbf{{\shobhika{\color{Saffron}\Huge ॐ}\\
 \vspace{50pt}
 ग॒णानां᳚ त्वा ग॒णप॑तिꣳ हवामहे
 क॒विं क॑वी॒नामु॑प॒मश्र॑वस्तमम् ।\\
 ज्ये॒ष्ठ॒राजं॒ ब्रह्म॑णां ब्रह्मणस्पत॒
 आन॑शृण्वन्नूतिभि॑स्सीद॒साद॑नम् ।।\\
 \vspace{50pt}
 गुरुर्ब्रह्मा गुरुर्विष्णुर्गुरुर्देवो महेश्वरः ।\\
 गुरुस्साक्षात् परब्रह्मा तस्मै श्री गुरवे नमः ।।\\
 \vspace{50pt}
 अज्ञान तिमिरान्धस्य ज्ञानाञ्जन शलाकया ।\\
 चक्षुुरुन्मीलितं येन तस्मै श्री गुरवे नमः ।।\\
 \vspace{50pt}
 अखण्ड मण्डलाकारं व्याप्तं येन चराचरम् ।\\
 तत्पदं दर्शितं येन तस्मै श्री गुरवे नमः ।।}}
\end{center}
\vspace*{\fill}
\newpage
\centering
\maketitle
\begin{tikzpicture}[overlay,remember picture]
    \draw [line width=1mm]
        ($ (current page.north west) + (1cm,-1cm) $)
        rectangle
        ($ (current page.south east) + (-1cm,1cm) $);
    \draw [line width=0.5mm]
        ($ (current page.north west) + (1.2cm,-1.2cm) $)
        rectangle
        ($ (current page.south east) + (-1.2cm,1.2cm) $);
\end{tikzpicture}
\vspace*{\fill}
\end{titlingpage}

%%%%%%%%%%%%%%%%%%%%%%%%%%%%%%%%%%%%%%%%%%%
\begin{center}
\textbf{\color{Saffron}\Large अनुक्रमणिका}
\end{center}
\tableofcontents

\newpage
%%%%%%%%%%%%%%%%%%%%%%%%%%%%%%%%%%%%%%%%%%%

\pagenumbering{arabic}

%%%%%%%%%%%%%%%%%%%%%%%%%%%%%%%%%%%%%%%%%%%%%%%%{Sarala Varases}%%%%%%%%%%%%%%%%%%%%%%%%%%%%%%%%%%%%%%%%%%%%%%%%%%%

\section{सरल वरसे}

%\begin{center}
% \textbf{सरल वरसे}
%\end{center}

\begin{center}
\begin{tabular*}{\textwidth}{l @{\extracolsep{\fill}} r}
रागम् : मायामाळवगौळ & तालम् : आदि  \\
\end{tabular*}
\end{center}

\vspace{20pt}
१.

\begin{center}
\begin{longtable}{ @{\extracolsep{\fill}} c c c c c c c c c c c c }
 ।। & स & रि & ग & म & । & प & द & । & नि & $\tar{स}$ & ।। \\
 \\
 ।। & $\tar{स}$ & नि & द & प & । & म & ग & । & रि & स & ।। 
\end{longtable}
\end{center}

\vspace{20pt}
२.

\begin{center}
\begin{longtable}{ @{\extracolsep{\fill}} c c c c c c c c c c c c }
 ।। & स & रि & ग & म & । & स & रि & । & ग & म & ।। \\
 \\
 ।। & स & रि & ग & म & । & प & द & । & नि & $\tar{स}$ & ।। \\
 \\
 ।। & $\tar{स}$ & नि & द & प & । & $\tar{स}$ & नि & । & द & प & ।। \\
 \\
 ।। & $\tar{स}$ & नि & द & प & । & म & ग & । & रि & स & ।। \\
\end{longtable}
\end{center}

\vspace{20pt}
३.

\begin{center}
\begin{longtable}{ @{\extracolsep{\fill}} c c c c c c c c c c c c }
 ।। & स & रि & स & रि & । & स & रि & । & ग & म & ।। \\
 \\
 ।। & स & रि & ग & म & । & प & द & । & नि & $\tar{स}$ & ।। \\
 \\
 ।। & $\tar{स}$ & नि & $\tar{स}$& नि & । & $\tar{स}$ & नि & । & द & प & ।। \\
 \\
 ।। & $\tar{स}$ & नि & द & प & । & म & ग & । & रि & स & ।। \\
\end{longtable}
\end{center}

\newpage
\vspace{20pt}
४.

\begin{center}
\begin{longtable}{ @{\extracolsep{\fill}} c c c c c c c c c c c c }
 ।। & स & रि & ग & म & । & प & म & । & ग & रि & ।। \\
 \\
 ।। & स & रि & ग & म & । & प & द & । & नि & $\tar{स}$ & ।। \\
 \\
 ।। & $\tar{स}$& नि & द & प & । & म & प & । & द & नि & ।। \\
 \\
 ।। & $\tar{स}$& नि & द & प & । & म & ग & । & रि & स & ।। \\
\end{longtable}
\end{center}


\vspace{20pt}
५.

\begin{center}
\begin{longtable}{ @{\extracolsep{\fill}} c c c c c c c c c c c c }
 ।। & स & रि & ग & म & । & प & म & । & द & प & ।। \\
 \\
 ।। & स & रि & ग & म & । & प & द & । & नि & $\tar{स}$& ।। \\
 \\
 ।। & $\tar{स}$& नि & द & प & । & म & प & । & ग & म & ।। \\
 \\
 ।। & $\tar{स}$& नि & द & प & । & म & ग & । & रि & स & ।। \\
\end{longtable}
\end{center}

\vspace{20pt}
६.

\begin{center}
\begin{longtable}{ @{\extracolsep{\fill}} c c c c c c c c c c c c }
 ।। & स & रि & ग & स & । & रि & ग & । & स & रि & ।। \\
 \\
 ।। & स & रि & ग & म & । & प & द & । & नि &$\tar{स}$ & ।। \\
 \\
 ।। & $\tar{स}$& नि & द & $\tar{स}$& । & नि & द & । & $\tar{स}$& नि & ।। \\
 \\
 ।। & $\tar{स}$& नि & द & प & । & म & ग & । & रि & स & ।। \\
\end{longtable}
\end{center}

\newpage

%%%%%%%%%%%%%%%%%%%%%%%%%%%%%%%%%%%%%%%%%%%%%%%%{Deergha Varases}%%%%%%%%%%%%%%%%%%%%%%%%%%%%%%%%%%%%%%%%%%%%%%%%%%%

\section{दीर्घ वरसे}

%\begin{center}
% \textbf{दीर्घ वरसे}
%\end{center}

\begin{center}
\begin{tabular*}{\textwidth}{l @{\extracolsep{\fill}} r}
रागम् : मायामाळवगौळ & तालम् : आदि  \\
\end{tabular*}
\end{center}

\vspace{20pt}
१.

\begin{center}
\begin{longtable}{ @{\extracolsep{\fill}} c c c c c c c c c c c c }
 ।। & स & रि & ग & म & । & प & , & । & प & , & ।। \\
 \\
 ।। & स & रि & ग & म & । & प & द & । & नि & $\tar{स}$& ।। \\
 \\
 ।। & $\tar{स}$& नि & द & प & । & म & , & । & म & , & ।। \\
 \\
 ।। & $\tar{स}$& नि & द & प & । & म & ग & । & रि & स & ।। \\
\end{longtable}
\end{center}

\vspace{20pt}
२.

\begin{center}
\begin{longtable}{ @{\extracolsep{\fill}} c c c c c c c c c c c c }
 ।। & स & रि & ग & म & । & प & , & । & स & , & ।। \\
 \\
 ।। & स & रि & ग & म & । & प & द & । & नि & $\tar{स}$ & ।। \\
 \\
 ।। & $\tar{स}$& नि & द & प & । & म & , & । & $\tar{स}$& , & ।। \\
 \\
 ।। & $\tar{स}$& नि & द & प & । & म & ग & । & रि & स & ।। \\
\end{longtable}
\end{center}

\vspace{20pt}
३.

\begin{center}
\begin{longtable}{ @{\extracolsep{\fill}} c c c c c c c c c c c c }
 ।। & स & रि & ग & म & । & प & , & । & स & रि & ।। \\
 \\
 ।। & स & रि & ग & म & । & प & द & । & नि & $\tar{स}$ & ।। \\
 \\
 ।। & $\tar{स}$& नि & द & प & । & म & , & । & $\tar{स}$ & नि & ।। \\
 \\
 ।। & $\tar{स}$& नि & द & प & । & म & ग & । & रि & स & ।। \\
\end{longtable}
\end{center}

\vspace{20pt}
४.

\begin{center}
\begin{longtable}{ @{\extracolsep{\fill}} c c c c c c c c c c c c }
 ।। & स & रि & ग & म & । & प & , & । & द & नि & ।। \\
 \\
 ।। & स & रि & ग & म & । & प & द & । & नि & $\tar{स}$& ।। \\
 \\
 ।। & $\tar{स}$& नि & द & प & । & म & , & । & ग & रि & ।। \\
 \\
 ।। & $\tar{स}$& नि & द & प & । & म & ग & । & रि & स & ।। \\
\end{longtable}
\end{center}

\newpage

%%%%%%%%%%%%%%%%%%%%%%%%%%%%%%%%%%%%%%%%%%%%%%%%{Dhattu Varases}%%%%%%%%%%%%%%%%%%%%%%%%%%%%%%%%%%%%%%%%%%%%%%%%%%%

\section{दाट्टु वरसे}

%\begin{center}
% \textbf{दाट्टु वरसे}
%\end{center}

\begin{center}
\begin{tabular*}{\textwidth}{l @{\extracolsep{\fill}} r}
रागम् : मायामाळवगौळ & तालम् : आदि  \\
\end{tabular*}
\end{center}

\vspace{20pt}
१.

\begin{center}
\begin{longtable}{ @{\extracolsep{\fill}} c c c c c c c c c c c c }
 ।। & स & ग & रि & म & । & ग & प & । & म & द & ।। \\
 \\
 ।। & स & रि & ग & म & । & प & द & । & नि & $\tar{स}$& ।। \\
 \\
 ।। & $\tar{स}$& द & नि & प & । & द & म & । & प & ग & ।। \\
 \\
 ।। & $\tar{स}$& नि & द & प & । & म & ग & । & रि & स & ।। \\
\end{longtable}
\end{center}

\vspace{20pt}
२.

\begin{center}
\begin{longtable}{ @{\extracolsep{\fill}} c c c c c c c c c c c c }
 ।। & स & म & रि & प & । & ग & द & । & म & नि & ।। \\
 \\
 ।। & स & रि & ग & म & । & प & द & । & नि & $\tar{स}$& ।। \\
 \\
 ।। & $\tar{स}$& प & नि & म & । & द & ग & । & प & रि & ।। \\
 \\
 ।। & $\tar{स}$& नि & द & प & । & म & ग & । & रि & स & ।। \\
\end{longtable}
\end{center}

\vspace{20pt}
३.

\begin{center}
\begin{longtable}{ @{\extracolsep{\fill}} c c c c c c c c c c c c }
 ।। & स & प & रि & द & । & ग & नि & । & म & स & ।। \\
 \\
 ।। & स & रि & ग & म & । & प & द & । & नि & $\tar{स}$ & ।। \\
 \\
 ।। & $\tar{स}$& म & नि & ग & । & द & रि & । & प & स & ।। \\
 \\
 ।। & $\tar{स}$& नि & द & प & । & म & ग & । & रि & स & ।। \\
\end{longtable}
\end{center}

\vspace{20pt}
४.

\begin{center}
\begin{longtable}{ @{\extracolsep{\fill}} c c c c c c c c c c c c }
 ।। & स & ग & प & नि & । & रि & म & । & द & $\tar{स}$& ।। \\
 \\
 ।। & स & रि & ग & म & । & प & द & । & नि & $\tar{स}$ & ।। \\
 \\
 ।। & $\tar{स}$& द & म & रि & । & नि & प & । & ग & स & ।। \\
 \\
 ।। & $\tar{स}$& नि & द & प & । & म & ग & । & रि & स & ।। \\
\end{longtable}
\end{center}

\newpage

%%%%%%%%%%%%%%%%%%%%%%%%%%%%%%%%%%%%%%%%%%%%%%%%{Mandhara Sthayi}%%%%%%%%%%%%%%%%%%%%%%%%%%%%%%%%%%%%%%%%%%%%%%%%%%%

\section{मन्दर स्थायि वरसे}

%\begin{center}
% \textbf{मन्दर स्थायि वरसे}
%\end{center}

\begin{center}
\begin{tabular*}{\textwidth}{l @{\extracolsep{\fill}} r}
रागम् : मायामाळवगौळ & तालम् : आदि  \\
\end{tabular*}
\end{center}

\vspace{20pt}
१.

\begin{center}
\begin{longtable}{ @{\extracolsep{\fill}} c c c c c c c c c c c c }
 ।। & $\tar{स}$& नि & द & प & । & म & ग & । & रि & स & ।। \\
 \\
 ।। & स & , & स & , & । & स & , & । & स & , & ।। \\
 \\
 ।। & स &  $\man{नि}$ & स & रि & । & स & रि & । & ग & म & ।। \\
 \\
 ।। & स & रि & ग & म & । & प & द & । & नि & $\tar{स}$& ।। \\
\end{longtable}
\end{center}

\vspace{20pt}
२.

\begin{center}
\begin{longtable}{ @{\extracolsep{\fill}} c c c c c c c c c c c c }
 ।। & $\tar{स}$ & नि & द & प & । & म & ग & । & रि & स & ।। \\
 \\
 ।। & स & , & स & , & । & स & , & । & स & , & ।। \\
 \\
 ।। & स & $\man{नि}$ & $\man{द}$ & $\man{नि}$ & । & स & रि & । & ग & म & ।। \\
 \\
 ।। & प & म & ग & रि & । & स & र & । & स & $\man{नि}$ & ।। \\
 \\
 ।। & स &  $\man{नि}$ & स & रि & । & स & रि & । & ग & म & ।। \\
 \\
 ।। & स & रि & ग & म & । & प & द & । & नि & $\tar{स}$ & ।। \\
\end{longtable}
\end{center}


\vspace{20pt}
३.

\begin{center}
\begin{longtable}{ @{\extracolsep{\fill}} c c c c c c c c c c c c }
 ।। & $\tar{स}$ & नि & द & प & । & म & ग & । & रि & स & ।। \\
 \\
 ।। & स & , & स & , & । & स & , & । & स & , & ।। \\
 \\
 ।। & स & $\man{नि}$ & $\man{द}$ & $\man{प}$ & । &  $\man{द}$ & $\man{नि}$ & । & स & रि & ।। \\
 \\
 ।। & ग & म & प & म & । & ग & रि & । & स & $\man{नि}$ & ।। \\
 \\
 ।। & स & $\man{नि}$ & $\man{द}$ & $\man{नि}$ & । & स & रि & । & ग & म & ।। \\
 \\
 ।। & प & म & ग & रि & । & स & र & । & स & $\man{नि}$ & ।। \\
 \\
 ।। & स &  $\man{नि}$ & स & रि & । & स & रि & । & ग & म & ।। \\
 \\
 ।। & स & रि & ग & म & । & प & द & । & नि & $\tar{स}$ & ।। \\
\end{longtable}
\end{center}

\newpage

%%%%%%%%%%%%%%%%%%%%%%%%%%%%%%%%%%%%%%%%%%%%%%%%{Tara Sthayi}%%%%%%%%%%%%%%%%%%%%%%%%%%%%%%%%%%%%%%%%%%%%%%%%%%%

\section{तार स्थायि वरसे}

%\begin{center}
% \textbf{तार स्थायि वरसे}
%\end{center}

\begin{center}
\begin{tabular*}{\textwidth}{l @{\extracolsep{\fill}} r}
रागम् : मायामाळवगौळ & तालम् : आदि  \\
\end{tabular*}
\end{center}

\vspace{20pt}
१.

\begin{center}
\begin{longtable}{ @{\extracolsep{\fill}} c c c c c c c c c c c c }
 ।। & स & रि & ग & म & । & प & द & । & नि & $\tar{स}$ & ।। \\
 \\
 ।। & $\tar{स}$ & , & $\tar{स}$& , & । & $\tar{स}$ & , & । & $\tar{स}$ & , & ।। \\
 \\
 ।। & द & नि & $\tar{स}$ & $\tar{रि}$ & । & $\tar{स}$ & नि & । & द & प & ।। \\
 \\
 ।। & $\tar{स}$ & नि & द & प & । & म & ग & । & रि & स & ।। \\
\end{longtable}
\end{center}

\vspace{20pt}
२.

\begin{center}
\begin{longtable}{ @{\extracolsep{\fill}} c c c c c c c c c c c c }
 ।। & स & रि & ग & म & । & प & द & । & नि & $\tar{स}$ & ।। \\
 \\
 ।। & $\tar{स}$ & , & $\tar{स}$& , & । & $\tar{स}$ & , & । & $\tar{स}$ & , & ।। \\
 \\
 ।। & द & नि & $\tar{स}$ & $\tar{रि}$ & । & $\tar{स}$ & $\tar{स}$ & । & $\tar{रि}$ & $\tar{स}$ & ।। \\
 \\
 ।। & $\tar{स}$ & $\tar{रि}$ & $\tar{स}$ & नि & । & द & प & । & म & प & ।। \\
 \\
 ।। & द & नि & $\tar{स}$ & $\tar{रि}$ & । & $\tar{स}$ & नि & । & द & प & ।। \\
 \\
 ।। & $\tar{स}$ & नि & द & प & । & म & ग & । & रि & स & ।। \\
\end{longtable}
\end{center}

\vspace{20pt}
३.

\begin{center}
\begin{longtable}{ @{\extracolsep{\fill}} c c c c c c c c c c c c }
 ।। & स & रि & ग & म & । & प & द & । & नि & $\tar{स}$ & ।। \\
 \\
 ।। & $\tar{स}$ & , & $\tar{स}$& , & । & $\tar{स}$ & , & । & $\tar{स}$ & , & ।। \\
 \\
 ।। & द & नि & $\tar{स}$ & $\tar{रि}$ & । & $\tar{ग}$ & $\tar{रि}$ & । & $\tar{स}$ & $\tar{रि}$ & ।। \\
 \\
 ।। & $\tar{स}$ & $\tar{रि}$ & $\tar{स}$ & नि & । & द & प & । & म & प & ।। \\
 \\
 ।। & द & नि & $\tar{स}$ & $\tar{रि}$ & । & $\tar{स}$ & $\tar{स}$ & । & $\tar{रि}$ & $\tar{स}$ & ।। \\
 \\
 ।। & $\tar{स}$ & $\tar{रि}$ & $\tar{स}$ & नि & । & द & प & । & म & प & ।। \\
 \\
 ।। & द & नि & $\tar{स}$ & $\tar{रि}$ & । & $\tar{स}$ & नि & । & द & प & ।। \\
 \\
 ।। & $\tar{स}$ & नि & द & प & । & म & ग & । & रि & स & ।। \\
\end{longtable}
\end{center}


\vspace{20pt}
४.

\begin{center}
\begin{longtable}{ @{\extracolsep{\fill}} c c c c c c c c c c c c }
 ।। & स & रि & ग & म & । & प & द & । & नि & $\tar{स}$ & ।। \\
 \\
 ।। & $\tar{स}$ & , & $\tar{स}$& , & । & $\tar{स}$ & , & । & $\tar{स}$ & , & ।। \\
 \\
 ।। & द & नि & $\tar{स}$ & $\tar{रि}$ & । & $\tar{ग}$ & $\tar{म}$ & । & $\tar{ग}$ & $\tar{रि}$ & ।। \\
 \\
 ।। & $\tar{स}$ & $\tar{रि}$ & $\tar{स}$ & नि & । & द & प & । & म & प & ।। \\
 \\
 ।। & द & नि & $\tar{स}$ & $\tar{रि}$ & । & $\tar{ग}$ & $\tar{रि}$ & । & $\tar{स}$ & $\tar{रि}$ & ।। \\
 \\
 ।। & $\tar{स}$ & $\tar{रि}$ & $\tar{स}$ & नि & । & द & प & । & म & प & ।। \\
 \\
 ।। & द & नि & $\tar{स}$ & $\tar{रि}$ & । & $\tar{स}$ & $\tar{स}$ & । & $\tar{रि}$ & $\tar{स}$ & ।। \\
 \\
 ।। & $\tar{स}$ & $\tar{रि}$ & $\tar{स}$ & नि & । & द & प & । & म & प & ।। \\
 \\
 ।। & द & नि & $\tar{स}$ & $\tar{रि}$ & । & $\tar{स}$ & नि & । & द & प & ।। \\
 \\
 ।। & $\tar{स}$ & नि & द & प & । & म & ग & । & रि & स & ।। \\
\end{longtable}
\end{center}

\newpage

%%%%%%%%%%%%%%%%%%%%%%%%%%%%%%%%%%%%%%%%%%%%%%%%{Jhanti Varases}%%%%%%%%%%%%%%%%%%%%%%%%%%%%%%%%%%%%%%%%%%%%%%%%%%%

\section{झण्टि वरसे}

%\begin{center}
% \textbf{झण्टि वरसे}
%\end{center}

\begin{center}
\begin{tabular*}{\textwidth}{l @{\extracolsep{\fill}} r}
रागम् : मायामाळवगौळ & तालम् : आदि  \\
\end{tabular*}
\end{center}

\vspace{20pt}
१.

\begin{center}
\begin{tabular*}{\textwidth}{ @{\extracolsep{\fill}} c c c c c c c c c c c c }
 ।। & स स & रि रि & ग ग & म म & । & प प & द द & । & नि नि & $\tar{स}\tar{स}$ & ।। \\
 \\
 ।। & $\tar{स}\tar{स}$ & नि नि & द द & प प & । & म म & ग ग & । & रि रि & स स & ।। 
\end{tabular*}
\end{center}

\vspace{20pt}
२.

\begin{center}
\begin{longtable}{ @{\extracolsep{\fill}} c c c c c c c c c c c c }
 ।। & स स & रि रि & ग ग & म म & । & रि रि & ग ग & । & म म & प प & ।। \\
 \\
 ।। & ग ग & म म & प प & द द & । & म म & प प & । & द द & नि नि & ।। \\
 \\
 ।। & प प & द द & नि नि & $\tar{स}\tar{स}$ & । & $\tar{स}\tar{स}$ & नि नि & । & द द & प प & ।। \\
 \\
 ।। & नि नि & द द & प प & म म & । & द द & प प & । & म म & ग ग & ।। \\
 \\
 ।। & प प & म म & ग ग & रि रि & । & म म & ग ग & । & रि रि & स स & ।। \\
\end{longtable}
\end{center}

\vspace{20pt}
३.

\begin{center}
\begin{longtable}{ @{\extracolsep{\fill}} c c c c c c c c c c c c }
 ।। & स स & रि रि & ग ग & रि रि & । & स स & रि रि & । & ग ग & म म & ।। \\
 \\
 ।। & रि रि & ग ग & म म & ग ग & । & रि रि & ग ग & । & म म & प प & ।। \\
 \\
 ।। & ग ग & म म & प प & म म & । & ग ग & म म & । & प प & द द & ।। \\
 \\
 ।। & म म & प प & द द & प प & । & म म & प प & । & द द & नि नि & ।। \\
 \\
 ।। & प प & द द & नि नि & द द & । & प प & द द & । & नि नि & $\tar{स}\tar{स}$ & ।। \\
 \\
 ।। & $\tar{स}\tar{स}$ & नि नि & द द & नि नि & । & $\tar{स}\tar{स}$ & नि नि & । & द द & प प & ।। \\
 \\
 ।। & नि नि & द द & प प & द द & । & नि नि & द द & । & प प & म म & ।। \\ 
 \\
 ।। & द द & प प & म म & प प & । & द द & प प & । & म म & ग ग & ।। \\ 
 \\
 ।। & प प & म म & ग ग & म म & । & प प & म म & । & ग ग & रि रि & ।। \\ 
 \\
 ।। & म म & ग ग & रि रि & ग ग & । & म म & ग ग & । & रि रि & स स & ।। \\ 
\end{longtable}
\end{center}

\vspace{20pt}
४.

\begin{center}
\begin{longtable}{ @{\extracolsep{\fill}} c c c c c c c c c c c c }
 ।। & स स & म म & ग ग & रि रि & । & स स & रि रि & । & ग ग & म म & ।। \\
 \\
 ।। & रि रि & प प & म म & ग ग & । & रि रि & ग ग & । & म म & प प & ।। \\
 \\
 ।। & ग ग & द द & प प & म म & । & ग ग & म म & । & प प & द द & ।। \\
 \\
 ।। & म म & नि नि & द द & प प & । & म म & प प & । & द द & नि नि & ।। \\
 \\
 ।। & प प & $\tar{स}\tar{स}$ & नि नि & द द & । & प प & द द & । & नि नि & $\tar{स}\tar{स}$ & ।। \\
 \\
 ।। & $\tar{स}\tar{स}$ & प प & द द & नि नि & । & $\tar{स}\tar{स}$ & नि नि & । & द द & प प & ।। \\
 \\
 ।। & नि नि & म म & प प & द द & । & नि नि & द द & । & प प & म म & ।। \\ 
 \\
 ।। & द द & ग ग & म म & प प & । & द द & प प & । & म म & ग ग & ।। \\ 
 \\
 ।। & प प & रि रि & ग ग & म म & । & प प & म म & । & ग ग & रि रि & ।। \\ 
 \\
 ।। & म म & स स & रि रि & ग ग & । & म म & ग ग & । & रि रि & स स & ।। \\ 
\end{longtable}
\end{center}

\vspace{20pt}
५.

\begin{center}
\begin{longtable}{ @{\extracolsep{\fill}} c c c c c c c c c c c c }
 ।। & स स & ग ग & रि रि & म म & । & स स & रि रि & । & ग ग & म म & ।। \\
 \\
 ।। & रि रि & म म & ग ग & प प & । & रि रि & ग ग & । & म म & प प & ।। \\
 \\
 ।। & ग ग & प प & म म & द द & । & ग ग & म म & । & प प & द द & ।। \\
 \\
 ।। & म म & द द & प प & नि नि & । & म म & प प & । & द द & नि नि & ।। \\
 \\
 ।। & प प & नि नि & द द & $\tar{स}\tar{स}$ & । & प प & द द & । & नि नि & $\tar{स}\tar{स}$ & ।। \\
 \\
 ।। & $\tar{स}\tar{स}$ & द द & नि नि & प प & । & $\tar{स}\tar{स}$ & नि नि & । & द द & प प & ।। \\
 \\
 ।। & नि नि & प प & द द & म म & । & नि नि & द द & । & प प & म म & ।। \\ 
 \\
 ।। & द द & म म & प प & ग ग & । & द द & प प & । & म म & ग ग & ।। \\ 
 \\
 ।। & प प & ग ग & म म & रि रि & । & प प & म म & । & ग ग & रि रि & ।। \\ 
 \\
 ।। & म म & रि रि & ग ग & स स & । & म म & ग ग & । & रि रि & स स & ।। \\ 
\end{longtable}
\end{center}

\vspace{20pt}
६.

\begin{center}
\begin{longtable}{ @{\extracolsep{\fill}} c c c c c c c c c c c c }
 ।। & स स & रि स & स रि & स रि & । & स स & रि रि & । & ग ग & म म & ।। \\
 \\
 ।। & रि रि & ग रि & रि ग & रि ग & । & रि रि & ग ग & । & म म & प प & ।। \\
 \\
 ।। & ग ग & म ग & ग म & ग म & । & ग ग & म म & । & प प & द द & ।। \\
 \\
 ।। & म म & प म & म प & म प & । & म म & प प & । & द द & नि नि & ।। \\
 \\
 ।। & प प & द प & प द & प द & । & प प & द द & । & नि नि & $\tar{स}\tar{स}$ & ।। \\
 \\
 ।। & $\tar{स}\tar{स}$ & नि $\tar{स}$ & $\tar{स}$नि & $\tar{स}$नि & । & $\tar{स}\tar{स}$ & नि नि & । & द द & प प & ।। \\
 \\
 ।। & नि नि & द नि & नि द & नि द & । & नि नि & द द & । & प प & म म & ।। \\ 
 \\
 ।। & द द & प द & द प & द प & । & द द & प प & । & म म & ग ग & ।। \\ 
 \\
 ।। & प प & म प & प म & प म & । & प प & म म & । & ग ग & रि रि & ।। \\ 
 \\
 ।। & म म & ग म & म ग & म ग & । & म म & ग ग & । & रि रि & स स & ।। \\ 
\end{longtable}
\end{center}

\vspace{20pt}
७.

\begin{center}
\begin{longtable}{ @{\extracolsep{\fill}} c c c c c c c c c c c c }
 ।। & स स & रि रि & ग स & रि ग & । & स स & रि रि & । & ग ग & म म & ।। \\
 \\
 ।। & रि रि & ग ग & म रि & ग म & । & रि रि & ग ग & । & म म & प प & ।। \\
 \\
 ।। & ग ग & म म & प ग & म प & । & ग ग & म म & । & प प & द द & ।। \\
 \\
 ।। & म म & प प & द म & प द & । & म म & प प & । & द द & नि नि & ।। \\
 \\
 ।। & प प & द द & नि प & द नि & । & प प & द द & । & नि नि & $\tar{स}\tar{स}$ & ।। \\
 \\
 ।। & $\tar{स}\tar{स}$ & नि नि & द $\tar{स}$ & नि द & । & $\tar{स}\tar{स}$ & नि नि & । & द द & प प & ।। \\
 \\
 ।। & नि नि & द द & प नि & द प & । & नि नि & द द & । & प प & म म & ।। \\ 
 \\
 ।। & द द & प प & म द & प म & । & द द & प प & । & म म & ग ग & ।। \\ 
 \\
 ।। & प प & म म & ग प & म ग & । & प प & म म & । & ग ग & रि रि & ।। \\ 
 \\
 ।। & म म & ग ग & रि म & ग रि & । & म म & ग ग & । & रि रि & स स & ।। \\ 
\end{longtable}
\end{center}

\vspace{20pt}
७.

\begin{center}
\begin{longtable}{ @{\extracolsep{\fill}} c c c c c c c c c c c }
 ।। & स स स & रि रि रि & ग म & । & स स & रि रि & । & ग ग & म म & ।। \\
 \\
 ।। & रि रि रि & ग ग ग & म प & । & रि रि & ग ग & । & म म & प प & ।। \\
 \\
 ।। & ग ग ग & म म म & प द & । & ग ग & म म & । & प प & द द & ।। \\
 \\
 ।। & म म म & प प प & द नि & । & म म & प प & । & द द & नि नि & ।। \\
 \\
 ।। & प प प & द द द & नि $\tar{स}$ & । & प प & द द & । & नि नि & $\tar{स}\tar{स}$ & ।। \\
 \\
 ।। & $\tar{स}\tar{स}\tar{स}$ & नि नि नि & द प & । & $\tar{स}\tar{स}$ & नि नि & । & द द & प प & ।। \\
 \\
 ।। & नि नि नि & द द द & प म & । & नि नि & द द & । & प प & म म & ।। \\ 
 \\
 ।। & द द द & प प प & म ग & । & द द & प प & । & म म & ग ग & ।। \\ 
 \\
 ।। & प प प & म म म & ग रि & । & प प & म म & । & ग ग & रि रि & ।। \\ 
 \\
 ।। & म म म & ग ग ग & रि स & । & म म & ग ग & । & रि रि & स स & ।। \\ 
\end{longtable}
\end{center}

\newpage

%%%%%%%%%%%%%%%%%%%%%%%%%%%%%%%%%%%%%%%%%%%%%%%%{Music theory}%%%%%%%%%%%%%%%%%%%%%%%%%%%%%%%%%%%%%%%%%%%%%%%%%%%

\section{स्वर राग ताल शास्त्र}

%\begin{center}
% \textbf{स्वर-राग-ताल शास्त्र}
%\end{center}

\textbf{स्वराः}

\begin{itemize}
 \item षड्ज - प्रकृति
 \item रिषभ - शुद्ध (रि$_{\text{१}}$) \hspace{40pt} चतुःश्रुति (रि$_{\text{२}}$) \hspace{40pt} षड्श्रुति (रि$_{\text{३}}$)
 \item गान्धार - शुद्ध (ग$_{\text{१}}$) \hspace{40pt} साधारण (ग$_{\text{२}}$) \hspace{40pt} अन्तर (ग$_{\text{३}}$)
 \item मध्यम - शुद्ध (म$_{\text{१}}$) \hspace{40pt} प्रति (म$_{\text{२}}$)
 \item पञ्चम - प्रकृति
 \item धैवत - शुद्ध (द$_{\text{१}}$) \hspace{40pt} चतुःश्रुति (द$_{\text{२}}$) \hspace{40pt} षड्श्रुति (द$_{\text{३}}$)
 \item निषाद - शुद्ध (नि$_{\text{१}}$) \hspace{40pt} कैशिकी (नि$_{\text{२}}$) \hspace{40pt} काकली (नि$_{\text{३}}$)
\end{itemize}
\newpage

%%%%%%%%%%%%%%%%%%%%%%%%%%%%%%%%%%%%%%%%%%%%%%{Alankarams}%%%%%%%%%%%%%%%%%%%%%%%%%%%%%%%%%%%%%%%%%

\section{अलङ्काराः}

%\begin{center}
% \textbf{अलङ्काराः (रागम् : मायामाळवगौळ)}
%\end{center}

\begin{center}
 १. चतुरश्र जाति ध्रुव ताल (\textbf{ । ० । । })
\end{center}

\begin{center}
\begin{longtable}{ @{\extracolsep{\fill}} c c c c c c c c c c c c c c c c c c c}
 ।। & स & रि & ग & म & । & ग & रि & । & स & रि & ग & रि & । & स & रि & ग & म & ।। \\
 \\
 ।। & रि & ग & म & प & । & म & ग & । & रि & ग & म & ग & । & रि & ग & म & प & ।। \\
 \\
 ।। & ग & म & प & द & । & प & म & । & ग & म & प & म & । & ग & म & प & द & ।। \\
 \\
 ।। & म & प & द & नि & । & द & प & । & म & प & द & प & । & म & प & द & नि & ।। \\
 \\
 ।। & प & द & नि & $\tar{स}$ & । & नि & द & । & प & द & नि & द & । & प & द & नि & $\tar{स}$ & ।। \\
 \\
 ।। & $\tar{स}$ & नि & द & प & । & द & नि & । & $\tar{स}$ & नि & द & नि & । & $\tar{स}$ & नि & द & प & ।। \\
 \\
 ।। & नि & द & प & म & । & प & द & । & नि & द & प & द & । & नि & द & प & म & ।। \\
 \\
 ।। & द & प & म & ग & । & म & प & । & द & प & म & प & । & द & प & म & ग & ।। \\
 \\
 ।। & प & म & ग & रि & । & ग & म & । & प & म & ग & म & । & प & म & ग & रि & ।। \\
 \\
 ।। & म & ग & रि & स & । & रि & ग & । & म & ग & रि & ग & । & म & ग & रि & स & ।।  
\end{longtable}
\end{center}

\vspace{20pt}

\begin{center}
 २. चतुरश्र जाति मठ्य ताल (\textbf{ । ० । })
\end{center}

\begin{center}
\begin{longtable}{ @{\extracolsep{\fill}} c c c c c c c c c c c c c c}
 ।। & स & रि & ग & रि & । & स & रि & । & स & रि & ग & म & ।। \\
 \\
 ।। & रि & ग & म & ग & । & रि & ग & । & रि & ग & म & प & ।। \\
 \\
 ।। & ग & म & प & म & । & ग & म & । & ग & म & प & द & ।। \\
 \\
 ।। & म & प & द & प & । & म & प & । & म & प & द & नि & ।। \\
 \\
 ।। & प & द & नि & द & । & प & द & । & प & द & नि & $\tar{स}$ & ।। \\
 \\
 ।। & $\tar{स}$ & नि & द & नि & । & $\tar{स}$ & नि & । & $\tar{स}$ & नि & द & प & ।। \\
 \\
 ।। & नि & द & प & द & । & नि & द & । & नि & द & प & म & ।। \\
 \\
 ।। & द & प & म & प & । & द & प & । & द & प & म & ग & ।। \\
 \\
 ।। & प & म & ग & म & । & प & म & । & प & म & ग & रि & ।। \\
 \\
 ।। & म & ग & रि & ग & । & म & ग & । & म & ग & रि & स & ।।  
\end{longtable}
\end{center}

\vspace{20pt}

\begin{center}
 ३. चतुरश्र जाति रूपक ताल (\textbf{ ० । })
\end{center}

\begin{center}
\begin{longtable}{ @{\extracolsep{\fill}} c c c c c c c c c}
 ।। & स & रि & । & स & रि & ग & म & ।। \\
 \\
 ।। & रि & ग & । & रि & ग & म & प & ।। \\
 \\
 ।। & ग & म & । & ग & म & प & द & ।। \\
 \\
 ।। & म & प & । & म & प & द & नि & ।। \\
 \\
 ।। & प & द & । & प & द & नि & $\tar{स}$ & ।। \\
 \\
 ।। & $\tar{स}$ & नि & । & $\tar{स}$ & नि & द & प & ।। \\
 \\
 ।। & नि & द & । & नि & द & प & म & ।। \\
 \\
 ।। & द & प & । & द & प & म & ग & ।। \\
 \\
 ।। & प & म & । & प & म & ग & रि & ।। \\
 \\
 ।। & म & ग & । & म & ग & रि & स & ।।  
\end{longtable}
\end{center}

\vspace{20pt}

\begin{center}
 ४. मिश्र जाति झम्प ताल (\textbf{। \begin{tikzpicture} \draw [line width=0.4mm] (.5ex,0) arc[start angle=180, end angle=360, radius=.8ex]; \end{tikzpicture} ०})
\end{center}

\begin{center}
\begin{longtable}{ @{\extracolsep{\fill}} c c c c c c c c c c c c c c}
 ।। & स & रि & ग & स & रि & स & रि & । & ग & । & म & , & ।। \\
 \\
 ।। & रि & ग & म & रि & ग & रि & ग & । & म & । & प & , & ।। \\
 \\
 ।। & ग & म & प & ग & म & ग & म & । & प & । & द & , & ।। \\
 \\
 ।। & म & प & द & म & प & म & प & । & द & । & नि & , & ।। \\
 \\
 ।। & प & द & नि & प & द & प & द & । & नि & । & $\tar{स}$ & , & ।। \\
 \\
 ।। & $\tar{स}$ & नि & द & $\tar{स}$ & नि & $\tar{स}$ & नि & । & द & । & प & , & ।। \\
 \\
 ।। & नि & द & प & नि & द & नि & द & । & प & । & म & , & ।। \\
 \\
 ।। & द & प & म & द & प & द & प & । & म & । & ग & , & ।। \\
 \\
 ।। & प & म & ग & प & म & प & म & । & ग & । & रि & , & ।। \\
 \\
 ।। & म & ग & रि & म & ग & म & ग & । & रि & । & स & , & ।। \\
\end{longtable}
\end{center}

\vspace{20pt}

\begin{center}
 ५. तिश्र जाति त्रिपुट ताल (\textbf{ । ० ० })
\end{center}

\begin{center}
\begin{longtable}{ @{\extracolsep{\fill}} c c c c c c c c c c c}
 ।। & स & रि & ग & । & स & रि & । & ग & म & ।। \\
 \\
 ।। & रि & ग & म & । & रि & ग & । & म & प & ।। \\
 \\
 ।। & ग & म & प & । & ग & म & । & प & द & ।। \\
 \\
 ।। & म & प & द & । & म & प & । & द & नि & ।। \\
 \\
 ।। & प & द & नि & । & प & द & । & नि & $\tar{स}$ & ।। \\
 \\
 ।। & $\tar{स}$ & नि & द & । & $\tar{स}$ & नि & । & द & प & ।। \\
 \\
 ।। & नि & द & प & । & नि & द & । & प & म & ।। \\
 \\
 ।। & द & प & म & । & द & प & । & म & ग & ।। \\
 \\
 ।। & प & म & ग & । & प & म & । & ग & रि & ।। \\
 \\
 ।। & म & ग & रि & । & म & ग & । & रि & स & ।।  
\end{longtable}
\end{center}

\vspace{20pt}

\begin{center}
 ६. खण्ड जाति अट्ट ताल (\textbf{ । । ० ० })
\end{center}

\begin{center}
\begin{longtable}{ @{\extracolsep{\fill}} c c c c c c c c c c c c c c c c c c c}
 ।। & स & रि & , & ग & , & । & स & , & रि & ग & , & । & म & , & । & म & , &।। \\
 \\
 ।। & रि & ग & , & म & , & । & रि & , & ग & म & , & । & प & , & । & प & , &।। \\
 \\
 ।। & ग & म & , & प & , & । & ग & , & म & प & , & । & द & , & । & द & , &।। \\
 \\
 ।। & म & प & , & द & , & । & म & , & प & द & , & । & नि & , & । & नि & , &।। \\
 \\
 ।। & प & द & , & नि & , & । & प & , & द & नि & , & । & $\tar{स}$ & , & । & $\tar{स}$ & , &।। \\
 \\
 ।। & $\tar{स}$ & नि & , & द & , & । & $\tar{स}$ & , & नि & द & , & । & प & , & । & प & , &।। \\
 \\
 ।। & नि & द & , & प & , & । & नि & , & द & प & , & । & म & , & । & म & , &।। \\
 \\
 ।। & द & प & , & म & , & । & द & , & प & म & , & । & ग & , & । & ग & , &।। \\
 \\
 ।। & प & म & , & ग & , & । & प & , & म & ग & , & । & रि & , & । & रि & , &।। \\
 \\
 ।। & म & ग & , & रि & , & । & म & , & ग & रि & , & । & स & , & । & स & , &।। 
\end{longtable}
\end{center}

\vspace{20pt}

\begin{center}
 ५. चतुरश्र जाति एक ताल (\textbf{ । })
\end{center}

\begin{center}
\begin{longtable}{ @{\extracolsep{\fill}} c c c c c c c c c c c}
 ।। & स & रि & ग & म & ।। & रि & ग & म & प & ।। \\
 \\
 ।। & ग & म & प & द & ।। & म & प & द & नि & ।। \\
 \\
 ।। & प & द & नि & $\tar{स}$ & ।। & $\tar{स}$ & नि & द & प & ।। \\
 \\
 ।। & नि & द & प & म & ।। & द & प & म & ग & ।। \\
 \\
 ।। & प & म & ग & रि & ।। & म & ग & रि & स & ।।
\end{longtable}
\end{center}


\newpage

%%%%%%%%%%%%%%%%%%%%%%%%%%%%%%%%%%%%%% Pillari Geethas %%%%%%%%%%%%%%%%%%%%%%%%%%%%%%%%%%%%%

\section{पिळ्ळारि गीतानि}

%\begin{center}
% \large{\textbf{पिळ्ळारि गीतानि}}
%\end{center}

%%%%%%%%%%%%%%%%%%%%%%%%%%%%%%%%%%%%%%%%%%%%%%%%%%%%%%%%%%%%%%%%%%%%%%%%%%%%%%%%%%%%%%%%%%%%

\subsection{श्री गणनाथ}

%\begin{center}
% \textbf{१. श्री गणनाथ}
%\end{center}

\begin{center}
\begin{tabular*}{\textwidth}{l @{\extracolsep{\fill}} r}
रागम् : मलहरि \index[ragas]{मलहरि! श्री गणनाथ} & तालम् : रूपक  \\
आरोहणम् : स रि$_{\text{१}}$ म$_{\text{१}}$ प द$_{\text{१}}$ $\tar{स}$ & रचयिता : श्री पुरन्दर दास \index[composers]{श्री पुरन्दर दास! श्री गणनाथ}\\
अवरोहणम् : $\tar{स}$ द$_{\text{१}}$ प म$_{\text{१}}$ ग$_{\text{३}}$ रि$_{\text{१}}$ स & \\
\end{tabular*}
\end{center}

\begin{center}
\renewcommand*{\arraystretch}{1.5}
\begin{longtable}{*{15} c}
\hline
\hline
 ।। & म & प & द & $\tar{स}$ & $\tar{स}$ & $\tar{रि}$ & ।। & $\tar{रि}$ & $\tar{स}$ & द & प & म & प & ।। \\ 
 \rowcolor{Gray}
   & श्री &  & ग & ण & ना & थ & & सिन् & धू &  & र & व & र्ण & \\
 \rowcolor{Gray}
   & सिद् & ध & चा & & र & ण & & ग & ण & से & & वि & त & \\
 \rowcolor{Gray}
   & स & क & ल & वि & द्या & & & आ & दि & पू & & जि & त & \\
   \\
 ।। & रि & म & प & द & म & प & ।। & द & प & म & ग & रि & स & ।। \\
 \rowcolor{Gray}
   & क & रु & ण & सा & ग & रा & & क & रि & व & द & ना & & \\ 
   \rowcolor{Gray}
   & सिद् & धि & वि & ना & य & का & & ते & & न & मो & न & मो & \\
   \rowcolor{Gray}
   & सर् & & वो & & त्त & म & & ते & & न & मो & न & मो & \\
   \\
 ।। & स & रि & म & , & ग & रि & ।। & स & रि & ग & रि & स & , & ।। \\
 \rowcolor{Gray}
   & लम् & & बो & & द & र & & ल & कु & मि & क & र & & \\
   \\
 ।। & रि & म & प & द & म & प & ।। & द & प & म & ग & रि & स & ।। \\
 \rowcolor{Gray}
   & अम् & & बा & & सु & त & & अ & म & र & वि & नु & त & \\ 
\hline
\hline
\end{longtable}
\end{center}

\newpage
%%%%%%%%%%%%%%%%%%%%%%%%%%%%%%%%%%%%%%%%%%%%%%%%%%%%%%%%%%%%%%%%%%%%%%%%%%%%%%%%%%%%%%%%%%%%

\subsection{कुन्द गौर}
%\begin{center}
% \textbf{२. कुन्द गौर}
%\end{center}

\begin{center}
\begin{tabular*}{\textwidth}{l @{\extracolsep{\fill}} r}
रागम् : मलहरि \index[ragas]{मलहरि! कुन्द गौर} & तालम् : रूपक  \\
आरोहणम् : स रि$_{\text{१}}$ म$_{\text{१}}$ प द$_{\text{१}}$ $\tar{स}$ & रचयिता : श्री पुरन्दर दास \index[composers]{श्री पुरन्दर दास! कुन्द गौर}\\
अवरोहणम् : $\tar{स}$ द$_{\text{१}}$ प म$_{\text{१}}$ ग$_{\text{३}}$ रि$_{\text{१}}$ स & \\
\end{tabular*}
\end{center}

\begin{center}
\renewcommand*{\arraystretch}{1.5}
\begin{longtable}{*{15} c}
\hline
\hline
 ।। & द & प & म & ग & रि & स & ।। & रि & म & प & द & म & प & ।। \\ 
 \rowcolor{Gray}
   & कुन् & द & गौ &  &  & र & & गौ &  & रि &  & व & र & \\
 \rowcolor{Gray}
   & चन् & द & मा &  &  & म & & मन् &  & दा &  & कि & नि & \\
 \rowcolor{Gray}
   & हि & म & कू &  &  & ट & & सिम् &  & हा &  & स & न & \\
 ।। & द & $\tar{रि}$ & $\tar{रि}$ & $\tar{स}$ & द & प & ।। & द & प & म & ग & रि & स & ।। \\
 \rowcolor{Gray}
   & मन् & दि & रा & & & य & & मा &  & न & म & कु & ट & \\
 \rowcolor{Gray}
   & मन् & दि & रा & & & य & & मा &  & न & म & कु & ट & \\
 \rowcolor{Gray}
   & वि & रु & पा &  &  & क्ष & & क & रु & णा &  & क & र & \\ 
 ।। & स & , & रि & , & रि & , & ।। & द & प & म & ग & रि & स & ।। \\
 \rowcolor{Gray}
   & मन् &  & धा &  & रा &  & & कु & सु & मा &  & क & र & \\
 ।। & स & रि & प & म & ग & रि & ।। & स & रि & ग & रि & स &  , & ।। \\
 \rowcolor{Gray}
   & म & क & रन् &  & दं &  &  & व &  & सि & तु & रे &  & \\  
\hline
\hline
\end{longtable}
\end{center}
\newpage
%%%%%%%%%%%%%%%%%%%%%%%%%%%%%%%%%%%%%%%%%%%%%%%%%%%%%%%%%%%%%%%%%%%%%%%%%%%%%%
\subsection{केरेय नीरनु}
%\begin{center}
% \textbf{३. केरेय नीरनु}
%\end{center}

\begin{center}
\begin{tabular*}{\textwidth}{l @{\extracolsep{\fill}} r}
रागम् : मलहरि \index[ragas]{मलहरि! केरेय नीरनु} & तालम् : तिस्र त्रिपुट  \\
आरोहणम् : स रि$_{\text{१}}$ म$_{\text{१}}$ प द$_{\text{१}}$ $\tar{स}$ & रचयिता : श्री पुरन्दर दास \index[composers]{श्री पुरन्दर दास! केरेय नीरनु}\\
अवरोहणम् : $\tar{स}$ द$_{\text{१}}$ प म$_{\text{१}}$ ग$_{\text{३}}$ रि$_{\text{१}}$ स & \\
\end{tabular*}
\end{center}

\begin{center}
\renewcommand*{\arraystretch}{1.5}
\begin{longtable}{*{21} c}
\hline
\hline
 ।। & द & $\tar{स}$ & $\tar{स}$ & ।& द & प & । & म & प & ।। & द & द & प & । & म & म & । &  प & , & ।। \\ 
 \rowcolor{Gray}
 ।। & के & रे & य & ।& नी &  & । & र & नु & ।। & के & रे & गे & । & चेल् &  & । & लि &  & ।। \\
 \rowcolor{Gray}
 ।। & श्री &  & पु & ।& रन् &  & । & द & र & ।। & वि & ठ्ठ & ल & । & रा &  & । & य & न & ।। \\
 ।। & द & द & $\tar{स}$ & ।& द & प & । & म & प & ।। & द & द & प & । & म & ग & । &  रि & स & ।। \\ 
 \rowcolor{Gray}
 ।। & व & र & व & ।& प & डे& । & द & व & ।। & रन् &  & ते & । & का &  & । & णि & रो & ।। \\
 \rowcolor{Gray}
 ।। & च & र & ण & ।& क & म & । & ल & व & ।। & नम् &  & बि & । & ब & दु & । &  कि & रो & ।। \\
 ।। & स & रि & रि & ।& स & रि & । & स & रि & ।। & द & द & प & । & म & ग & । &  रि & स & ।। \\
 \rowcolor{Gray}
 ।। & ह & रि & य & ।& क & रु & । & ण & दो & ।। & ला &  & द & । & भ &  & । & ग्य & व & ।। \\
 ।। & द & प & द & ।& $\tar{स}$ & , & । & द & प & ।। & द & द & प & । & म & ग & । &  रि & स & ।। \\
 \rowcolor{Gray}
 ।। & ह & रि & स & ।& मर् &  & । & प & णे & ।। & मा &  & डि & । & ब & दु & । &  कि & रो & ।। \\
\hline
\hline
\end{longtable}
\end{center}
\newpage
%%%%%%%%%%%%%%%%%%%%%%%%%%%%%%%%%%%%%%%%%%%%%%%%%%%%%%%%%%%%%%%%%%%%%%%%%%%%%%
\subsection{पदुमनाभ}
%\begin{center}
% \textbf{४. पदुमनाभ}
%\end{center}

\begin{center}
\begin{tabular*}{\textwidth}{l @{\extracolsep{\fill}} r}
रागम् : मलहरि \index[ragas]{मलहरि! पदुमनाभ} & तालम् : तिस्र त्रिपुट  \\
आरोहणम् : स रि$_{\text{१}}$ म$_{\text{१}}$ प द$_{\text{१}}$ $\tar{स}$ & रचयिता : श्री पुरन्दर दास \index[composers]{श्री पुरन्दर दास! पदुमनाभ}\\
अवरोहणम् : $\tar{स}$ द$_{\text{१}}$ प म$_{\text{१}}$ ग$_{\text{३}}$ रि$_{\text{१}}$ स & \\
\end{tabular*}
\end{center}

\begin{center}
\renewcommand*{\arraystretch}{1.4}
\begin{longtable}{*{21} c}
\hline
\hline
 ।। & रि & स & $\man{द}$ & ।& स & , & । & स & , & ।। & म & ग & रि & । & म & म & । &  प & , & ।। \\ 
 \rowcolor{Gray}
 ।। & प & दु & म & ।& ना &  & । & भ &  & ।। & प & र & म & । & पु & रु & । & ष &  & ।। \\
 \rowcolor{Gray}
 ।। & वि & दु & र & ।& वन् &  & । & द्य &  & ।। & वि & म & ल & । & च & रि & । & त &  & ।। \\
 ।। & स & द & , & ।& द & प & । & म & प & ।। & द & द & प & । & म & ग & । &  रि & स & ।। \\ 
 \rowcolor{Gray}
 ।। & प & रञ् &  & ।& ज्यो & & । &  & ति & ।। & स्व & रु &  & । & प &  & । &  &  & ।। \\
 \rowcolor{Gray}
 ।। & वि & हङ् &  & ।& गा &  & । &  & व & ।। & रो &  & ह & । & ण &  & । &  &  & ।। \\
 ।। & प & म & प & ।& द & $\tar{स}$ & । & द & $\tar{स}$ & ।। & $\tar{रि}$ & $\tar{स}$ & द & । & द & $\tar{स}$ & । & द & प & ।। \\
 \rowcolor{Gray}
 ।। & उ & द & धि & ।& नि & वा & । &  & स & ।। & उ & र & ग & । & श & य & । & न &  & ।। \\
 ।। & द & द & प & ।& प & , & । & प & म & ।। & रि & म & म & । & प &  , & । & प &  , & ।। \\
 \rowcolor{Gray}
 ।। & उ &  & न्न & ।& तो &  & । & न्न & त & ।। & म & हि &  & । & मा &  & । &  &  & ।। \\
  ।। & द & द & प & ।& प & , & । & प & म & ।। & रि & रि & म & । & म & ग & । & रि & स & ।। \\
 \rowcolor{Gray}
 ।। & य & दु & कु & ।& लो &  & । & त्त & म & ।। & य &  & ज्ञ & । & र &  & । & क्ष & क & ।। \\
 ।। & स & , & स & ।& द & द & । & द & प & ।। & प & , & प & । & म & ग & । & रि & स & ।। \\
 \rowcolor{Gray}
 ।। & अा &  & ज्ञ & ।& शि &  & । & क्ष & क & ।। & रा &  & म & । & ना &  & । &  & म & ।। \\
 ।। & द & $\tar{स}$ & , & ।& द & प & । & म & प & ।। & द & द & प & । & म & ग & । &  रि & स & ।। \\ 
 \rowcolor{Gray}
 ।। & वि & भी &  & ।& ष & ण & । & प &  & ।। & ल & क &  & । & न & मो & । & न & मो & ।। \\
  ।। & द & $\tar{स}$ & , & ।& द & प & । & म & प & ।। & द & द & प & । & म & ग & । &  रि & स & ।। \\ 
 \rowcolor{Gray}
 ।। & वि & भो &  & ।& व & र & । & दा &  & ।। & य & क &  & । & न & मो & । & न & मो & ।। \\
।। & प & म & प & ।& द & $\tar{स}$ & । & द & $\tar{स}$ & ।। & $\tar{रि}$ & $\tar{स}$ & द & । & द & $\tar{स}$ & । & द & प & ।। \\
 \rowcolor{Gray}
 ।। & शु & भ &  & ।& प्र & द & । & सु & म & ।। & नो &  & र & । & द &  & । &  & सु & ।। \\
 ।। & द & द & प & ।& प & , & । & प & म & ।। & रि & म & म & । & प &  , & । & प &  , & ।। \\
 \rowcolor{Gray}
 ।। & रे &  & न्द्र & ।& म &  & । & नो &  & ।। & रञ् &  & ज & । & न &  & । &  &  & ।। \\
 ।। & द & द & प & ।& प & , & । & प & म & ।। & रि & रि & म & । & म & ग & । & रि & स & ।। \\
 \rowcolor{Gray}
 ।। & अ & भि &  & ।& न &  & । & व & पु & ।। & रन् &  & द & । & र &  & । &  & & ।। \\
 ।। & स & , & स & ।& द & द & । & द & प & ।। & प & , & प & । & म & ग & । & रि & स & ।। \\
 \rowcolor{Gray}
 ।। & वि & ठ्ठ & ल & ।& भल् &  & । & ल & रे & ।। & रा &  & म & । & ना &  & । &  & म & ।। \\
\hline
\hline
\end{longtable}
\end{center}
\newpage
%%%%%%%%%%%%%%%%%%%%%%%%%%%%%%%%%%Sanchari Geetham%%%%%%%%%%%%%%%%%%%%%%%%%%%%%%%%%%%%%%%%%%%%

\section{सञ्चारि गीतानि}

%\begin{center}
% \large{\textbf{सञ्चारि गीतानि}}
%\end{center}

%%%%%%%%%%%%%%%%%%%%%%%%%%%%%%%%%%%%%%%%%%%%%%%%%%%%%%%%%%%%%%%%%%%%%%%%%%%%%%%%%%%%%%%%%%%%
\subsection{वरवीणा}
%\begin{center}
% \textbf{१. वरवीणा}
%\end{center}

\begin{center}
\begin{tabular*}{\textwidth}{l @{\extracolsep{\fill}} r}
रागम् : मोहनम् \index[ragas]{मोहनम्! वरवीणा } & तालम् : रूपकम्  \\
आरोहणम् : स रि$_{\text{२}}$ ग$_{\text{३}}$ प द$_{\text{२}}$ $\tar{स}$ & रचयिता : श्री अप्पैय दीक्षितर् \index[composers]{श्री अप्पैय दीक्षितर्! वरवीणा}\\
अवरोहणम् : $\tar{स}$ द$_{\text{२}}$ प ग$_{\text{३}}$ रि$_{\text{२}}$ स & \\
\end{tabular*}
\end{center}



\begin{center}
\renewcommand*{\arraystretch}{1.5}
\begin{longtable}{ *{15} c}
\hline
\hline
 ।। & ग & ग & प & , & प & , & ।। & द & प & $\tar{स}$ & , & $\tar{स}$ & , & ।। \\ 
 \rowcolor{Gray}
   & व & र & वी &  & णा &  & & मृ & दु & पा &  & णी &  & \\
 ।। & $\tar{रि}$ & $\tar{स}$ & द & द & प & , & ।। & द & प & ग & ग & रि & , & ।। \\
 \rowcolor{Gray}
   & व & न & रु & ह & लो &  & & च & न & रा &  & णी & & \\ 
 ।। & ग & प & द & $\tar{स}$ & द & , & ।।  & द & प & ग & ग & रि & , & ।। \\
 \rowcolor{Gray}
   & सु & रु & चि & र & बम् &  & & ब & र & वे &  & णी & & \\
 ।। & ग & ग & द & प & ग & रि & ।। & प & ग & ग & रि & स & , & ।। \\
 \rowcolor{Gray}
   & सु & र & नु & त & कल् &  & & या &  &  &  & णी &  & \\ 
 ।। & ग & ग & ग & ग & रि & ग & ।। & प & ग & प & , & प & , & ।। \\
 \rowcolor{Gray}
   & नि & रु & प & म & शु & भ & & गु & ण & लो &  & ला &  & \\
 ।। & ग & ग & द & प & द &  , & ।। & द & प & $\tar{स}$ & , & $\tar{स}$ & , & ।। \\
 \rowcolor{Gray}
   & नि & र & त & ज & या &  & & प्र & द & शी &  & ला &  & \\
 ।। & द & $\tar{ग}$ & $\tar{रि}$ & $\tar{ग}$ & $\tar{रि}$ &  $\tar{स}$ & ।। & $\tar{रि}$ & $\tar{स}$ & द & $\tar{स}$ & द &  प & ।। \\
 \rowcolor{Gray}
   & व & र & दा &  & प्रि & य & & र & ङ्ग & ना &  & य & की & \\
 ।। & ग & प & द & $\tar{स}$ & द &  प & ।। & द & प & ग & ग & रि &  स & ।। \\
 \rowcolor{Gray}
   & वा &  & ञ्चि & त & फ & ल & & दा &  &  &  & य & की & \\
 ।। & स & रि & ग & , & ग & , & ।। & ग & रि & प & ग & रि & , & ।। \\
 \rowcolor{Gray}
   & स & र & सि &  & जा &  & & स & न & ज & न & नी &  & \\
 ।। & स & रि & स & ग & रि & स & ।। & & & & & & &\\
  \rowcolor{Gray}
   & ज & य & ज & य & ज & य & & & & & & & & \\ 
\hline
\hline
\end{longtable}
\end{center}
\newpage
%%%%%%%%%%%%%%%%%%%%%%%%%%%%%%%%%%%%%%%%%%%%%%%%%%%%%%%%%%%%%%%%%%%%%%%%%%%%%%
\subsection{जय जय जय}
%\begin{center}
% \textbf{२. जय जय जय}
%\end{center}

\begin{center}
\begin{tabular*}{\textwidth}{l @{\extracolsep{\fill}} r}
रागम् : बेगड \index[ragas]{बेगड! जय जय जय} & तालम् : तिस्र त्रिपुट  \\
आरोहणम् : स ग$_{\text{३}}$ रि$_{\text{२}}$ ग$_{\text{३}}$ म$_{\text{१}}$ प द$_{\text{२}}$ प $\tar{स}$ & रचयिता : * \index[composers]{*! जय जय जय}\\
अवरोहणम् : $\tar{स}$ नि$_{\text{३}}$ द$_{\text{२}}$ प म$_{\text{१}}$ ग$_{\text{३}}$ रि$_{\text{२}}$ स & \\
\end{tabular*}
\end{center}

\begin{center}
\renewcommand*{\arraystretch}{1.5}
\begin{longtable}{ *{21} c}
\hline
\hline
 ।। & म & ग & म & ।& प & , & । & द & प & ।। & नि$_{\text{२}}$ & द & प & । & म & ग & । &  म & प & ।। \\ 
 \rowcolor{Gray}
 ।। & ज & य & ज & ।& य &  & । & ज & य & ।। & वि & ज & य & । & वि & नु & । &  & ता & ।। \\
 ।। & $\tar{स}$ & नि & $\tar{स}$ & ।& द & प & । & $\tar{स}$ & , & ।। & $\tar{स}$ & नि & $\tar{स}$ & । & $\tar{रि}$ & $\tar{स}$ & । & $\tar{म}$ & $\tar{ग}$ & ।। \\ 
 \rowcolor{Gray}
 ।। & वि & म & ल & ।& च & रि & । & ता &  & ।। & वि & नु & त & । & हि & त & । & को &  & ।। \\
 ।। & $\tar{रि}$ & $\tar{स}$ & नि & ।& $\tar{स}$ & $\tar{रि}$ & । & $\tar{स}$ & नि & ।। & $\tar{स}$ & नि & $\tar{स}$ & । & द & नि & । & प & द & ।। \\ 
 \rowcolor{Gray}
 ।। & कि & ल &  & ।& ली &  & । & ला &  & ।। & अ &  &  & । &  &  & । &  &  & ।। \\
 ।। & नि$_{\text{२}}$ & द & प & ।& म & ग & । & रि & स & ।। & म & ग & म & । & रि & ग & । & म & प & ।। \\ 
 \rowcolor{Gray}
 ।। & अ &  &  & ।&  &  & । &  &  & ।। & म & धु & रे & । & मी &  & । & ना &  & ।। \\
 ।। & $\tar{स}$ & नि & $\tar{स}$ & ।& द & प & । & $\tar{स}$ & , & ।। &  &  &  &  &  &  &  &  &  & \\ 
 \rowcolor{Gray}
 ।। &  &  &  & ।&  &  & । & क्षी &  & ।। &  &  &  &  &  &  &  &  &  &  \\ 
 \hline
\hline
\end{longtable}
\end{center}
\newpage
%%%%%%%%%%%%%%%%%%%%%%%%%%%%%%%%%%%%%%%%%%%%%%%%%%%%%%%%%%%%%%%%%%%%%%%%%%%%%%

\subsection{मन्दर धारे}
%\begin{center}
% \textbf{३. मन्दर धारे}
%\end{center}

\begin{center}
\begin{tabular*}{\textwidth}{l @{\extracolsep{\fill}} r}
रागम् :  काम्भोजि \index[ragas]{ काम्भोजि! मन्दर धारे} & तालम् : आदि \\
आरोहणम् : स रि$_{\text{२}}$ ग$_{\text{३}}$ म$_{\text{१}}$ प द$_{\text{२}}$ $\tar{स}$ & रचयिता : श्री पैदल गुरुमूर्ति शास्त्री \index[composers]{श्री पैदल गुरुमूर्ति शास्त्री! मन्दर धारे}\\
अवरोहणम् : $\tar{स}$ नि$_{\text{२}}$ द$_{\text{२}}$ प म$_{\text{१}}$ ग$_{\text{३}}$ रि$_{\text{२}}$ स  $\man{नि}$$_{\text{३}}$ $\man{प}$ $\man{द}$$_{\text{२}}$ स & \\
\end{tabular*}
\end{center}

\begin{center}
\renewcommand*{\arraystretch}{1.4}
\begin{longtable}{ *{22} c}
\hline
\hline
 ।। & $\tar{स}$ & , & नि$_{\text{३}}$ & प & । & द & द & । & $\tar{स}$ & , & ।। \\
 \rowcolor{Gray}
 ।। & मन् &  & द & र & । & धा &  & । & रे & &  ।।\\
।। & द & $\tar{स}$ & $\tar{रि}$ & $\tar{म}$ & । & $\tar{ग}$ & $\tar{ग}$ & । & $\tar{रि}$ & $\tar{स}$ & ।। \\
 \rowcolor{Gray}
।। & मो &  & क्ष & मु & । & रा &  & । &  & रे & ।।\\
  ।। & $\tar{स}$ & $\tar{रि}$ & $\tar{स}$ & $\tar{स}$ & । & नि & नि & । & द & प & ।। \\
 \rowcolor{Gray}
 ।। & दै &  & त्य & सं & । & हा &  & । &  & रे & ।।\\
   ।। & द & द & प & म & । & ग & म & । & प & , & ।। \\
 \rowcolor{Gray}
 ।। & पा &  & व & न & । & मू &  & । & र्ते &  & ।।\\
 ।। & ग & प & द & $\tar{स}$ & । & नि & नि & । & द &  प & ।। \\
 \rowcolor{Gray}
 ।। & अ & र & वि & न्द & । & न & य & । & न &  & ।।\\
  ।। & द & द & प & म & । & ग & ग & । & रि &  स & ।। \\
 \rowcolor{Gray}
 ।। & व & ट & प & त्र & । & श & य & । & न &  & ।।\\
   ।। & ग & प & प & द & । & द & $\tar{स}$ & । & $\tar{स}$ & $\tar{रि}$ & ।। \\
 \rowcolor{Gray}
 ।। & अ &  &  &  & । & अ &  & । &  &  & ।।\\
 ।। & $\tar{रि}$ & $\tar{प}$ & $\tar{म}$ & $\tar{ग}$ & । & $\tar{रि}$ & $\tar{ग}$ & । & $\tar{रि}$ & $\tar{स}$ & ।।\\
 \rowcolor{Gray}
 ।। & अ &  &  &  & । & अ &  & । &  &  & ।।\\
  ।। & $\tar{स}$ & $\tar{रि}$ & $\tar{स}$ & $\tar{स}$ & । & नि & नि & । & द & प & ।। \\
 \rowcolor{Gray}
 ।। & दै &  & त्य & सं & । & हा &  & । &  & रे & ।।\\
   ।। & द & द & प & म & । & ग & म & । & प & , & ।। \\
 \rowcolor{Gray}
 ।। & पा &  & व & न & । & मू &  & । & र्ते &  & ।।\\
 । & ग & प & द & $\tar{स}$ & । & नि & नि & । & द &  प & ।। \\
 \rowcolor{Gray}
 ।। & प & द & शु & भ & । & रे &  & । &  & ख & ।।\\
  ।। & द & द & प & म & । & ग & ग & । & रि &  स & ।। \\
 \rowcolor{Gray}
 ।। & म & कु & ट & म & । & यू &  & । &  & र & ।।\\
।। & $\tar{स}$ & , & नि$_{\text{३}}$ & प & । & द & द & । & $\tar{स}$ & & ।। \\
 \rowcolor{Gray}
 ।। & मन् &  & द & र & । & धा &  & । & रे & & ।।\\
 \hline
\hline
\end{longtable}
\end{center}

\newpage
%%%%%%%%%%%%%%%%%%%%%%%%%%%%%%%%%%%%%%%%%%%%%%%%%%%%%%%%%%%%%%%%%%%%%%%%%%%%%%
\subsection{कमलजा दल}
%\begin{center}
% \textbf{४. कमलजा दल}
%\end{center}

\begin{center}
\begin{tabular*}{\textwidth}{l @{\extracolsep{\fill}} r}
रागम् : कल्याणी \index[ragas]{कल्याणी! कमलजा दल} & तालम् : तिश्र त्रिपुट \\
आरोहणम् : स रि$_{\text{२}}$ ग$_{\text{३}}$ म$_{\text{२}}$ प द$_{\text{२}}$ नि$_{\text{३}}$ $\tar{स}$ & रचयिता :  श्री पुरन्दर दास\index[composers]{श्री पुरन्दर दास! कमलजा दल}\\
अवरोहणम् : $\tar{स}$ नि$_{\text{३}}$ द$_{\text{२}}$ प म$_{\text{२}}$ ग$_{\text{३}}$ रि$_{\text{२}}$ स  \\
\end{tabular*}
\end{center}

\begin{center}
\renewcommand*{\arraystretch}{1.5}
\begin{longtable}{ *{21} c}
\hline
\hline
 ।। & $\tar{स}$ & $\tar{स}$ & $\tar{स}$ & ।& नि & द & । & नि & $\tar{स}$ & ।। & नि & द & प & । & द & प & । &  म & प & ।। \\ 
 \rowcolor{Gray}
 ।। & क & म & ल & ।& जा &  & । & द & ल & ।। & वि & म & ल & । & सु & न & । & य & न & ।। \\
 ।। & ग & म & प & ।& प & द & । & द & नि & ।। & द & प & म & । & प & ग & । & रि & स & ।। \\ 
 \rowcolor{Gray}
 ।। & क & रि & व & ।& र & द & । & क & रु & ।। & णाम् &  & बु & । & धे &  & । & &  & ।। \\
 ।। & $\man{द}$ & $\man{नि}$ & $\man{द}$ & ।& ग & रि & । & ग & , & ।। & म & प & , & । & म & ग & । & रि & स & ।। \\ 
 \rowcolor{Gray}
 ।। & क & रु & ण & ।& शा & र & । & दे &  & ।। & क & म &  & । & ला &  & । &  &  & ।। \\
 ।। & रि & स & रि & ।& स & , & । & स & , & ।। & ग & म & प & । & म & प & । & द & प & ।। \\ 
 \rowcolor{Gray}
 ।। & कान् &  &  & ।& त &  & । &  &  & ।। & कम् &  & स & । & न & र & । & का &  & ।। \\
 ।। & नि & द & प & ।& द & प & । & म & प & ।। & ग & म & प & । & प & द & । & द & नि & ।।\\ 
 \rowcolor{Gray}
 ।। & सु & र & वि & ।& भे &  & । & द & न & ।। & व & र & द & । & वे & & । & ला &  & ।। \\ 
।। & द & प & म & ।& प & ग & । & रि & स & ।। & $\man{द}$ & $\man{नि}$ & $\man{द}$ & । & ग & रि & । & ग & , & ।।\\ 
 \rowcolor{Gray}
 ।। & पु & र & सु & ।& रोत् &  & । & त & म & ।। & क & रु & ण & । & शा & र & । & दे &  & ।। \\
 ।। & म & प & , & । & म & ग & । & रि & स & ।। & रि & स & रि & ।& स & , & । & स & , & ।।\\
 \rowcolor{Gray}
 ।। & क & म &  & । & ला &  & । &  &  & ।। & कान् &  &  & ।& त &  & । &  &  & ।।\\
 \hline
\hline
\end{longtable}
\end{center}
\newpage
%%%%%%%%%%%%%%%%%%%%%%%%%%%%%%%%%%%%%%%%%%%%%%%%%%%%%%%%%%%%%%%%%%%%%%%%%%%%%%

\subsection{कमल सुलोचन}
%\begin{center}
% \textbf{५. कमल सुलोचन}
%\end{center}

\begin{center}
\begin{tabular*}{\textwidth}{l @{\extracolsep{\fill}} r}
रागम् : आनन्द भैरवी \index[ragas]{आनन्द भैरवी! कमल सुलोचन} & तालम् : एक \\
आरोहणम् : स ग$_{\text{२}}$ रि$_{\text{२}}$ ग$_{\text{२}}$ म$_{\text{१}}$ प द$_{\text{२}}$ प $\tar{स}$ & रचयिता :  श्री पुरन्दर दास\index[composers]{श्री पुरन्दर दास! कमल सुलोचन}\\
अवरोहणम् : $\tar{स}$ नि$_{\text{२}}$ द$_{\text{२}}$ प म$_{\text{१}}$ ग$_{\text{२}}$ म$_{\text{१}}$ ग$_{\text{२}}$  रि$_{\text{२}}$ $\man{नि}$$_{\text{२}}$ स  \\
\end{tabular*}
\end{center}

\begin{center}
\renewcommand*{\arraystretch}{1.5}
%\setlength\LTleft{-0.3cm}
\begin{longtable}{ *{21} c}
\hline
\hline
 ।। & नि & द & नि & $\tar{स}$ & ।। & $\tar{स}$ & , & $\tar{स}$ & $\tar{स}$ & ।। & $\tar{ग}$ & $\tar{रि}$ & $\tar{स}$ & नि & ।। & नि & द & प & म & ।। \\
 \rowcolor{Gray}
 ।। & क & म & ल & सु & ।। & लो &  & च & न & ।। & वि & म & ल & त & ।। & टा &  & कि & नी & ।। \\ 
 ।। & प & प & , & द & ।। & नि & द & प & म & ।। & म & प & म & प & ।। & ग & रि & स & , & ।। \\
 \rowcolor{Gray}
 ।। & म & रा &  & ल & ।। & गा &  & मि & नी & ।। & क & रि & ह & र & ।। & म &  & ध्ये &  & ।।\\
  ।। & स & , & नि & , & ।। & स & ग & ग & म & ।। & ग & म & प & म & ।। & ग & , & रि & $\man{नि}$ & ।।\\
 \rowcolor{Gray}
 ।। & बि & म् & बा &  & ।। & द & रे & या &  & ।। & न & न & वि & दु & ।। & मण् &  & ड & ल & ।। \\

 ।। & स & , & स & , & ।। & & & & & & & & & & & & & & &\\
 \rowcolor{Gray}
 ।। & रे &  & रे &  & ।। & & & & & & & & & & & & & & & \\
  ।। & प & , & म & ग & ।। & म & , & ग & रि & ।। & ग & , & रि & $\man{नि}$ & ।। & स & , & स & , & ।। \\
 \rowcolor{Gray}
 ।। & च & न् & द & न & ।। & कु & ङ् & कु & म & ।। & प & ङ् & क & ज & ।। & रे &  & रे &  & ।। \\
 ।। & प & प & म & ग & ।। & म & म & ग & रि & ।। & ग & ग & रि & $\man{नि}$ & ।। & स & , & स & , & ।। \\
 \rowcolor{Gray}
 ।। & प & रि & म & ल & ।। & क & स् & तू & रि & ।। & ति & ल & क & द & ।। & रे &  & रे &  & ।। \\

 ।। & स & ग & रि & ग & ।। & म & ग & म & , & ।। & म & नि & द & नि & ।। & प & द & नि & $\tar{स}$ & ।।\\
 \rowcolor{Gray}
 ।। & जा &  & जि &  & ।। & शै &  & या & & ।। & क & च & कु & च & ।। & घ & न & ज & ग & ।। \\
 ।। & $\tar{ग}$ & $\tar{रि}$ & $\tar{स}$ & नि & ।। & नि & द & प & म & ।। & प & प & , & द & ।। & नि & द & प & म & ।।\\
 \rowcolor{Gray}
 ।। & ना &  & द & म् & ।। & बो &  & रु & ह & ।। & म & रा &  & ल & ।। & गा &  & मि & नी & ।। \\
 ।। & म & प & म & प & ।। & ग & रि & स & , & ।। & स & , & नि & , & ।। & स & ग & ग & म & ।। \\
 \rowcolor{Gray}
 ।। & क & रि & ह & र & ।। & म &  & ध्ये &  & ।। & बि & म् & बा &  & ।। & द & रे & या &  & ।। \\
 ।। & ग & म & प & म & ।। & ग & , & रि & $\man{नि}$ & ।।& स & , & स & , & ।। & & & & & \\
 \rowcolor{Gray}
 ।। & न & न & वि & दु & ।। & मण् &  & ड & ल & ।। & रे &  & रे &  & ।। & & & & & \\

 \hline
\hline
\end{longtable}
\end{center}

\newpage
%%%%%%%%%%%%%%%%%%%%%%%%%%%%%%%%%%%%%%%%%%%%%%%%%%%%%%%%%%%%%%%%%%%%%%%%%%%%%%
\section{स्वर जतयः}
\subsection{बिलहरि}
\begin{center}
\begin{tabular*}{\textwidth}{l @{\extracolsep{\fill}} r}
रागम् : बिलहरि \index[ragas]{बिलहरि! बिलहरि स्वरजति} & तालम् : आदि \\
आरोहणम् : स रि$_{\text{२}}$ ग$_{\text{३}}$ प द$_{\text{२}}$ प $\tar{स}$ & रचयिता :  \index[composers]{! बिलहरि स्वरजति}\\
अवरोहणम् : $\tar{स}$ नि$_{\text{३}}$ द$_{\text{२}}$ प म$_{\text{१}}$ ग$_{\text{३}}$ रि$_{\text{२}}$ स  \\
\end{tabular*}
\end{center}

\begin{center}
\renewcommand*{\arraystretch}{1.4}
\begin{longtable}{ *{20} c}
\hline
\hline
%----------------------------------------------------------
\multicolumn{20}{l}{ ।। पल्लवि ।।}\\
\\
।। & स & ,& ,& रि & ग & ,& प & , & । & द & ,& $\tar{स}$ & , & । & नि & , &  द & , & ।।\\
।। & प & , & द & प & म & ग & रि & स & । & रि & स & $\man{नि}$ & $\man{द}$ & । & स &  , &  , &  , & ।। \\ 
%----------------------------------------------------------
\\
\multicolumn{20}{l}{ ।। अनुपल्लवि ।।}\\
\\
।। & स & ,& ,& रि & ग & ,& प & , & । & म & ,& ग & , & । & प & , &  द & , & ।।\\
।। & $\tar{रि}$ & , & , & $\tar{स}$ & नि & , & , & द & । & प & , & , & म & । & ग &  , &  , &  रि & ।। \\
%----------------------------------------------------------
\\
\multicolumn{20}{l}{ ।। चरणानि ।।}\\
\\
१.\\
 ।। & ग & प & द & $\tar{रि}$ & $\tar{स}$ & $\tar{स}$ & $\tar{स}$ & , & । & $\tar{ग}$ & $\tar{ग}$ & $\tar{ग}$ & , & । & $\tar{रि}$ & $\tar{रि}$ &  $\tar{रि}$ & , & ।।\\
।। & प & प & प & , & म & ग & ग & , & । & रि & स & स & , & । & रि & स & $\man{नि}$ &  $\man{द}$ & ।। \\
\\
२. \\
 ।। & स & , & , & रि & ग & , & ग & , & । & ग & , & , & , & । & ग & , & रि  &  ग & ।।\\
।। & प & , & , & प & प & , & प & , & । & प & , & , & , & । & प & , & द &  प & ।। \\
।। & $\tar{स}$ & , & , & $\tar{स}$ & $\tar{स}$ & , & $\tar{स}$ & , & । & $\tar{ग}$ & $\tar{रि}$ & $\tar{स}$ & नि & । & नि & द &  प  & , & ।।\\
।। & प & द & प & म & ग & ग & रि & , & । & ग & प & म & ग & । & रि & स & रि  & ग & ।। \\
\\
३. \\
 ।। & प & प & प & , & रि & रि & रि & , & । & ग & प & म & ग & । & ग & , & ,  &  , & ।।\\
।। & ग & प & म & ग & म & ग  & रि & स & । & रि & स & रि & ग & । & स & , & , &  , & ।। \\
।। & रि & स & $\man{नि}$  & $\man{द}$ & स & , & , & , & । & म & ग & रि & ग & । & प & , &  ,  &  , & ।।\\
।। & द & प & द & $\tar{रि}$ & $\tar{स}$ & , & , & , & । & $\tar{रि}$ & $\tar{स}$ & नि & द & । & प & म & ग  & रि & ।। \\
\\
४.\\
।। & प & , & , & , & म & ग & रि & ग & । & द &  , &  , & , & । & म & ग & रि  & ग & ।।\\
।। & प & , & , & प & म & ग  & रि & ग & । & प & , & प & , & । & प & , & , &  , & ।। \\
।। & $\tar{ग}$ & , &  , & , & $\tar{रि}$ &  $\tar{स}$ & नि & द & । & $\tar{रि}$ & , & , & , & । & $\tar{रि}$ & $\tar{स}$ &  नि  &  द & ।।\\
।। & $\tar{स}$ & , & , & , & $\tar{रि}$ & $\tar{स}$ & नि & द & । & $\tar{स}$ & , & $\tar{स}$ & , & । & $\tar{स}$ &  , &  , &  , & ।। \\
।। & $\tar{ग}$ &  , & $\tar{रि}$ & स & $\tar{रि}$ & , & $\tar{रि}$ & , & । & $\tar{रि}$ &  , &  , & , & । & $\tar{रि}$ & , & $\tar{स}$  & नि & ।।\\
।। & द & , & द &  , & द & ,  & , & , & । & द & , & प & म & । & ग & , & ग &  , & ।। \\
।। & ग & , &  , & , & स &  रि & ग & द & । & प & , & , & , & । & $\tar{रि}$ & $\tar{स}$ &  $\tar{रि}$  &  $\tar{ग}$ & ।।\\
।। & $\tar{स}$ & , & , & , & $\tar{ग}$ & $\tar{रि}$ & $\tar{स}$ & नि & । & द & प & म & ग & । & रि & स & रि & ग & ।। \\
\\
५. \\
।। & द & , & , & , & प & द & प & म & । & ग & रि &  ग & , & । & ग &  , &  ,  &  , & ।।\\
।। & प &  द & प & म & ग & रि  & ग & , & । & ग & , & , & , & । & प & द & प & म & ।। \\
।। & ग & ग & रि & ग & स &  , & स & स & । & स & , & स & , & । & स & , &  ,  &  , & ।।\\
।। & स & रि & ग & , & रि & ग & प & , & । & द & प & द & , & । & म &  ग &  रि &  ग & ।। \\
।। & द &  , & प & म & ग & रि & स & , & । & स &  , &  , & , & । & $\tar{रि}$ & , & $\tar{स}$  & नि & ।।\\
।। & द & प & द &  , & द & ,  & , & , & । & $\tar{ग}$ & , & $\tar{रि}$ & $\tar{स}$ & । & नि & द & रि &  , & ।। \\
।। & $\tar{रि}$ & , &  , & , & $\tar{स}$ &  , & $\tar{रि}$ & $\tar{ग}$ & । & $\tar{स}$ & $\tar{रि}$ & $\tar{स}$ & नि & । & द & , &  , &  , & ।।\\
।। & प & , & द & नि & प & द & प & म & । & ग & , & ग & द & । & प & म & ग & रि & ।। \\
\hline
\hline
\end{longtable}
\end{center}
\newpage
%%%%%%%%%%%%%%%%%%%%%%%%%%%%%%%%%%%%%%%%%%%%%%%%%%%%%%%%%%%%%%%%%%%%%%%%%%%%%%
\subsection{खमास्}
\begin{center}
\begin{tabular*}{\textwidth}{l @{\extracolsep{\fill}} r}
रागम् : खमास् \index[ragas]{खमास्! खमास् स्वरजति} & तालम् : आदि \\
आरोहणम् : स म$_{\text{१}}$ ग$_{\text{३}}$ म$_{\text{१}}$ प द$_{\text{२}}$ नि$_{\text{२}}$ $\tar{स}$ & रचयिता :  \index[composers]{! खमास् स्वरजति}\\
अवरोहणम् : $\tar{स}$ नि$_{\text{२}}$ द$_{\text{२}}$ प म$_{\text{१}}$ ग$_{\text{३}}$ रि$_{\text{२}}$ स  \\
\end{tabular*}
\end{center}

%%%%%%%%%%%%%%%%%%%%%%%%%%%%%%%%%%%%%%%%%%%%%%%%%%%%%%%%%%%%%%%%%%%%%%%%%%%%%%

\printindex[ragas]
\printindex[composers]

%%%%%%%%%%%%%%%%%%%%%%%%%%%%%%%%%%%%%%%%%%%%%%%%%%%%%%%%%%%%%%%%%%%%%%%%%%%%%%
\end{sanskrit}

\end{document}
